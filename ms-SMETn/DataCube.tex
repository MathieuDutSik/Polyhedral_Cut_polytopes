\documentclass[12pt]{article}
\usepackage{graphicx,amssymb, amsmath, vmargin}
\usepackage[francais]{babel}
\usepackage[T1]{fontenc}
\usepackage[latin1]{inputenc}
\catcode`\�=\active
\catcode`\�=\active
\def�{\og\ignorespaces}%
\def�{{\fg}}%
\title{$s$-constant for $(m,s)$-supermetric in $n$-cube}
\setpapersize{custom}{21cm}{29.7cm}
\setmarginsrb{1cm}{3cm}{1cm}{1cm}{0pt}{0pt}{0pt}{0pt}

\newtheorem{prop}{Proposition}
\newtheorem{theor}{Theorem}
\newtheorem{cor}{Corollary}
\newtheorem{lem}{Lemma}
\newtheorem{claim}{Claim}
\newtheorem{conj}{Conjecture}
\newtheorem{definition}{Definition}
\newtheorem{remark}{remark}


\author{Mathieu Dutour}

\sloppy
\begin{document}
\newcommand{\R}{\ensuremath{\mathbb{R}}}
\newcommand{\N}{\ensuremath{\mathbb{N}}}
\newcommand{\Q}{\ensuremath{\mathbb{Q}}}
\newcommand{\C}{\ensuremath{\mathbb{C}}}
\newcommand{\Z}{\ensuremath{\mathbb{Z}}}
\newcommand{\T}{\ensuremath{\mathbb{T}}}

\maketitle




\section{The case $m=2$}

$m=2$, $n=3$, $s=\sqrt{3}$ attained for
\begin{equation*}
\begin{array}{ccc}
0&0&0\\ 
0&0&1\\ 
0&1&0\\ 
1&0&0
\end{array}
\end{equation*}



$m=2$, $n=4$, $s=\frac{1+2\sqrt{2}}{\sqrt{5}}$
\begin{equation*}
\begin{array}{cccc}
0&0&0&0\\ 
0&0&0&1\\ 
0&0&1&0\\ 
1&1&0&0
\end{array}
\end{equation*}


$m=2$, $n=5$, $s=\frac{1+2\sqrt{3}}{\sqrt{7}}$
\begin{equation*}
\begin{array}{ccccc}
0&0&0&0&0\\ 
0&0&0&0&1\\ 
0&0&0&1&0\\ 
1&1&1&0&0
\end{array}
\end{equation*}


$m=2$, $n=6$, $s=\frac{1+2\sqrt{4}}{\sqrt{9}}$
\begin{equation*}
\begin{array}{cccccc}
0&0&0&0&0&0\\ 
0&0&0&0&0&1\\ 
0&0&0&0&1&0\\ 
1&1&1&1&0&0
\end{array}
\end{equation*}



$m=2$, $n=7$, $s=\frac{1+2\sqrt{5}}{\sqrt{11}}$
\begin{equation*}
\begin{array}{ccccccc}
0&0&0&0&0&0&0\\ 
0&0&0&0&0&0&1\\ 
0&0&0&0&0&1&0\\ 
1&1&1&1&1&0&0
\end{array}
\end{equation*}


$m=2$, $n=8$, $s=\frac{1+2\sqrt{6}}{\sqrt{13}}$
\begin{equation*}
\begin{array}{cccccccc}
0&0&0&0&0&0&0&0\\ 
0&0&0&0&0&0&0&1\\ 
0&0&0&0&0&0&1&0\\ 
1&1&1&1&1&1&0&0
\end{array}
\end{equation*}



\section{The case $m=3$}

$m=3$, $n=4$, $s=\frac{1+3\sqrt{2}}{\sqrt{7}}$
\begin{equation*}
\begin{array}{cccc}
0&0&0&0\\ 
0&0&0&1\\ 
0&0&1&0\\ 
0&1&0&0\\ 
1&1&1&1
\end{array}
\end{equation*}


$m=3$, $n=5$, $s=\frac{1+3\sqrt{3}}{\sqrt{10}}$
\begin{equation*}
\begin{array}{ccccc}
0&0&0&0&0\\ 
0&0&0&0&1\\ 
0&0&0&1&0\\ 
0&0&1&0&0\\ 
1&1&1&1&1
\end{array}
\end{equation*}


$m=3$, $n=6$, $s=\frac{1+3\sqrt{4}}{\sqrt{13}}$
\begin{equation*}
\begin{array}{cccccc}
0&0&0&0&0&0\\ 
0&0&0&0&0&1\\ 
0&0&0&0&1&0\\ 
0&0&0&1&0&0\\ 
1&1&1&1&1&1
\end{array}
\end{equation*}


$m=3$, $n=7$, $s=\frac{1+3\sqrt{5}}{\sqrt{16}}$
\begin{equation*}
\begin{array}{ccccccc}
0&0&0&0&0&0&0\\ 
0&0&0&0&0&0&1\\ 
0&0&0&0&0&1&0\\ 
0&0&0&0&1&0&0\\ 
1&1&1&1&1&1&1
\end{array}
\end{equation*}




\section{The case $m=4$}

$m=4$, $n=5$, $s=\frac{1+4\sqrt{2}}{\sqrt{13}}$
\begin{equation*}
\begin{array}{ccccc}
0&0&0&0&0\\ 
0&0&0&0&1\\ 
0&0&0&1&0\\ 
0&0&1&0&0\\ 
0&1&0&0&0\\ 
1&1&1&1&1
\end{array}
\end{equation*}


$m=4$, $n=6$, $s=\frac{\sqrt{2}+\sqrt{3}+3\sqrt{4}}{\sqrt{25}}$
\begin{equation*}
\begin{array}{cccccc}
0&0&0&0&0&0\\ 
0&0&0&0&0&1\\ 
0&0&0&0&1&0\\ 
0&0&0&1&0&0\\ 
0&1&1&0&0&0\\ 
1&1&1&1&1&1
\end{array}
\end{equation*}


\section{The Icosahedron}
\noindent The set of vertices of the icosahedron $I$ is 
$(\pm(1+\sqrt(5)), 0, \pm 2)$, 
$(\pm 2, \pm(1+\sqrt(5)), 0)$, 
$(0, \pm 2, \pm(1+\sqrt(5)))$.

the s-constants are
\begin{equation*}
\begin{array}{c}
s_1(I)=\frac{4}{1+\sqrt{5}}\\[1cm]
s_2(I)=\frac{\sqrt{(6\sqrt{5}+30)}+2\sqrt{24\sqrt{5}+60}}{3\sqrt{5}+9}\\[1cm]
s_3(I)=\frac{2\sqrt{5}+3}{\sqrt{5}+2}
\end{array}
\end{equation*}


\section{The Dodecahedron}
\noindent The set of vertices of the dodecahedron $D$ is
$(\pm(4+2\sqrt{5}), \pm(1+\sqrt{5}), 0)$, 
$(\pm(1+\sqrt{5}), 0, \pm(4+2\sqrt{5}))$, 
$(0, \pm(4+2\sqrt{5}), \pm(1+\sqrt{5}))$, 
$(\pm(3+\sqrt{5}), \pm(3+\sqrt{5}), \pm(3+\sqrt{5}))$


the s-constants are
\begin{equation*}
\begin{array}{c}
s_1(D)=\sqrt{2}\frac{\sqrt{5}+2}{\sqrt{5}+3}\\[1cm]
s_2(D)=\sqrt{3}\frac{\sqrt{22\sqrt{5}+50}}{3\sqrt{5}+7}\\[1cm]
s_3(D)=\frac{27\sqrt{5}+61}{21\sqrt{5}+47}
\end{array}
\end{equation*}



\end{document}
