\documentclass[12pt]{article}
\usepackage{amsfonts, amsmath, latexsym, epsfig}
\usepackage{epsf}
\usepackage{url}
\title{Automorphism group of Lorentzian lattices}

\def\QuotS#1#2{\leavevmode\kern-.0em\raise.2ex\hbox{$#1$}\kern-.1em/\kern-.1em\lower.25ex\hbox{$#2$}}


%\usepackage{vmargin}
%\setpapersize{custom}{21cm}{29.7cm}
%\setmarginsrb{1.7cm}{1cm}{1.7cm}{3.5cm}{0pt}{0pt}{0pt}{0pt}
%marge gauche, marge haut, marge droite, marge bas.

\begin{document}
\newcommand{\R}{\ensuremath{\mathbb{R}}}
\newcommand{\N}{\ensuremath{\mathbb{N}}}
\newcommand{\Q}{\ensuremath{\mathbb{Q}}}
\newcommand{\C}{\ensuremath{\mathbb{C}}}
\newcommand{\Z}{\ensuremath{\mathbb{Z}}}
\newcommand{\T}{\ensuremath{\mathbb{T}}}
\newtheorem{proposition}{Proposition}
\newtheorem{theorem}{Theorem}
\newtheorem{corollary}{Corollary}
\newtheorem{lemma}{Lemma}
\newtheorem{problem}{Problem}
\newtheorem{conjecture}{Conjecture}
\newtheorem{claim}{Claim}
\newtheorem{remark}{Remark}
\newtheorem{definition}{Definition}
\newcommand{\qed}{\hfill $\Box$ }
\newcommand{\proof}{\noindent{\bf Proof.}\ \ }

Number of vertices $n=6$.

Adjacencies of Graph
\begin{enumerate}
\item vertex 1 adjacent to 2 4 5 6
\item vertex 2 adjacent to 1 3 5 6
\item vertex 3 adjacent to 2 4 5 6
\item vertex 4 adjacent to 1 3 5 6
\item vertex 5 adjacent to 1 2 3 4 6
\item vertex 6 adjacent to 1 2 3 4 5
\end{enumerate}
Size of automorphism group of the graph=16

Full group: $\vert Aut(polytope) \vert  =512$

Restricted group: $\vert Aut(G) \times switch \vert = 512$

Number of orbits for the full group : 4

List of orbits of facets for the full group:
Total number of orbits = 4
Total number of facets = 184
\begin{enumerate}
\item Inequality 1 with incidence 24 and stabilizer of size 16. Orbit size is 
32 nature: 3-cycle inequality, C=[ 2, 6, 1 ] F=[ 2, 6 ]
\begin{center}
\begin{tabular}{|c|c|c|c|c|c|}
\hline
(1,2) : 1    &    (1,4) : 0    &    (1,5) : 0    &    (1,6) : 1    &    (2,3) : 0    &    (2,5) : 0\\
(2,6) : -1    &    (3,4) : 0    &    (3,5) : 0    &    (3,6) : 0    &    (4,5) : 0    &    (4,6) : 0\\
(5,6) : 0    &                       &                       &                       &                       &                   \\
\hline
\end{tabular}
\end{center}
\item Inequality 2 with incidence 24 and stabilizer of size 32. Orbit size is 
16 nature: 3-cycle inequality, C=[ 5, 6, 1 ] F=[ 5, 6 ]
\begin{center}
\begin{tabular}{|c|c|c|c|c|c|}
\hline
(1,2) : 0    &    (1,4) : 0    &    (1,5) : 1    &    (1,6) : 1    &    (2,3) : 0    &    (2,5) : 0\\
(2,6) : 0    &    (3,4) : 0    &    (3,5) : 0    &    (3,6) : 0    &    (4,5) : 0    &    (4,6) : 0\\
(5,6) : -1    &                       &                       &                       &                       &                   \\
\hline
\end{tabular}
\end{center}
\item Inequality 3 with incidence 16 and stabilizer of size 64. Orbit size is 
8 nature: 4-cycle inequality, C=[ 1, 2, 3, 4 ] F=[ 1, 2 ]
\begin{center}
\begin{tabular}{|c|c|c|c|c|c|}
\hline
(1,2) : -1    &    (1,4) : 1    &    (1,5) : 0    &    (1,6) : 0    &    (2,3) : 1    &    (2,5) : 0\\
(2,6) : 0    &    (3,4) : 1    &    (3,5) : 0    &    (3,6) : 0    &    (4,5) : 0    &    (4,6) : 0\\
(5,6) : 0    &                       &                       &                       &                       &                   \\
\hline
\end{tabular}
\end{center}
\item Inequality 4 with incidence 14 and stabilizer of size 4. Orbit size is 128 nature: unknown
\begin{center}
\begin{tabular}{|c|c|c|c|c|c|}
\hline
(1,2) : -1    &    (1,4) : 1    &    (1,5) : 1    &    (1,6) : -1    &    (2,3) : 1    &    (2,5) : 0\\
(2,6) : 0    &    (3,4) : 1    &    (3,5) : 1    &    (3,6) : -1    &    (4,5) : -1    &    (4,6) : 1\\
(5,6) : 1    &                       &                       &                       &                       &                   \\
\hline
\end{tabular}
\end{center}
\end{enumerate}
\end{document}

