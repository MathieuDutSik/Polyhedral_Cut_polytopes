\documentclass[12pt]{article}
\usepackage{amsfonts, amsmath, latexsym, epsfig}
\usepackage{epsf}
\usepackage{url}
\title{Automorphism group of Lorentzian lattices}

\def\QuotS#1#2{\leavevmode\kern-.0em\raise.2ex\hbox{$#1$}\kern-.1em/\kern-.1em\lower.25ex\hbox{$#2$}}


%\usepackage{vmargin}
%\setpapersize{custom}{21cm}{29.7cm}
%\setmarginsrb{1.7cm}{1cm}{1.7cm}{3.5cm}{0pt}{0pt}{0pt}{0pt}
%marge gauche, marge haut, marge droite, marge bas.

\begin{document}
\newcommand{\R}{\ensuremath{\mathbb{R}}}
\newcommand{\N}{\ensuremath{\mathbb{N}}}
\newcommand{\Q}{\ensuremath{\mathbb{Q}}}
\newcommand{\C}{\ensuremath{\mathbb{C}}}
\newcommand{\Z}{\ensuremath{\mathbb{Z}}}
\newcommand{\T}{\ensuremath{\mathbb{T}}}
\newtheorem{proposition}{Proposition}
\newtheorem{theorem}{Theorem}
\newtheorem{corollary}{Corollary}
\newtheorem{lemma}{Lemma}
\newtheorem{problem}{Problem}
\newtheorem{conjecture}{Conjecture}
\newtheorem{claim}{Claim}
\newtheorem{remark}{Remark}
\newtheorem{definition}{Definition}
\newcommand{\qed}{\hfill $\Box$ }
\newcommand{\proof}{\noindent{\bf Proof.}\ \ }

Number of vertices $n=9$.

Adjacencies of Graph
\begin{enumerate}
\item vertex 1 adjacent to 6 7 8 9
\item vertex 2 adjacent to 6 7 8 9
\item vertex 3 adjacent to 6 7 8 9
\item vertex 4 adjacent to 6 7 8 9
\item vertex 5 adjacent to 6 7 8 9
\item vertex 6 adjacent to 1 2 3 4 5 7 8 9
\item vertex 7 adjacent to 1 2 3 4 5 6 8 9
\item vertex 8 adjacent to 1 2 3 4 5 6 7 9
\item vertex 9 adjacent to 1 2 3 4 5 6 7 8
\end{enumerate}
Size of automorphism group of the graph=2880

Full group: $\vert Aut(polytope) \vert  =737280$

Restricted group: $\vert Aut(G) \times switch \vert = 737280$

Number of orbits for the full group : 4

List of orbits of facets for the full group:
Total number of orbits = 4
Total number of facets = 1496
\begin{enumerate}
\item Inequality 1 with incidence 192 and stabilizer of size 6144. Orbit size is 
120 nature: 3-cycle inequality, C=[ 1, 6, 9 ] F=[ 1, 6 ]
\begin{center}
\begin{tabular}{|c|c|c|c|c|c|}
\hline
(1,6) : -1    &    (1,7) : 0    &    (1,8) : 0    &    (1,9) : 1    &    (2,6) : 0    &    (2,7) : 0\\
(2,8) : 0    &    (2,9) : 0    &    (3,6) : 0    &    (3,7) : 0    &    (3,8) : 0    &    (3,9) : 0\\
(4,6) : 0    &    (4,7) : 0    &    (4,8) : 0    &    (4,9) : 0    &    (5,6) : 0    &    (5,7) : 0\\
(5,8) : 0    &    (5,9) : 0    &    (6,7) : 0    &    (6,8) : 0    &    (6,9) : 1    &    (7,8) : 0\\
(7,9) : 0    &    (8,9) : 0    &                       &                       &                       &                   \\
\hline
\end{tabular}
\end{center}
\item Inequality 2 with incidence 192 and stabilizer of size 46080. Orbit size is 
16 nature: 3-cycle inequality, C=[ 6, 7, 8 ] F=[ 6, 7 ]
\begin{center}
\begin{tabular}{|c|c|c|c|c|c|}
\hline
(1,6) : 0    &    (1,7) : 0    &    (1,8) : 0    &    (1,9) : 0    &    (2,6) : 0    &    (2,7) : 0\\
(2,8) : 0    &    (2,9) : 0    &    (3,6) : 0    &    (3,7) : 0    &    (3,8) : 0    &    (3,9) : 0\\
(4,6) : 0    &    (4,7) : 0    &    (4,8) : 0    &    (4,9) : 0    &    (5,6) : 0    &    (5,7) : 0\\
(5,8) : 0    &    (5,9) : 0    &    (6,7) : -1    &    (6,8) : 1    &    (6,9) : 0    &    (7,8) : 1\\
(7,9) : 0    &    (8,9) : 0    &                       &                       &                       &                   \\
\hline
\end{tabular}
\end{center}
\item Inequality 3 with incidence 160 and stabilizer of size 9216. Orbit size is 
80 nature: Hypermetric, b=[ 0, 0, 1, 0, 0, 1, 1, -1, -1 ]
\begin{center}
\begin{tabular}{|c|c|c|c|c|c|}
\hline
(1,6) : 0    &    (1,7) : 0    &    (1,8) : 0    &    (1,9) : 0    &    (2,6) : 0    &    (2,7) : 0\\
(2,8) : 0    &    (2,9) : 0    &    (3,6) : -1    &    (3,7) : -1    &    (3,8) : 1    &    (3,9) : 1\\
(4,6) : 0    &    (4,7) : 0    &    (4,8) : 0    &    (4,9) : 0    &    (5,6) : 0    &    (5,7) : 0\\
(5,8) : 0    &    (5,9) : 0    &    (6,7) : -1    &    (6,8) : 1    &    (6,9) : 1    &    (7,8) : 1\\
(7,9) : 1    &    (8,9) : -1    &                       &                       &                       &                   \\
\hline
\end{tabular}
\end{center}
\item Inequality 4 with incidence 112 and stabilizer of size 576. Orbit size is 1280 nature: unknown
\begin{center}
\begin{tabular}{|c|c|c|c|c|c|}
\hline
(1,6) : 0    &    (1,7) : 0    &    (1,8) : 0    &    (1,9) : 0    &    (2,6) : 0    &    (2,7) : 0\\
(2,8) : 0    &    (2,9) : 0    &    (3,6) : 1    &    (3,7) : 1    &    (3,8) : -1    &    (3,9) : -1\\
(4,6) : -1    &    (4,7) : 1    &    (4,8) : 1    &    (4,9) : 1    &    (5,6) : 0    &    (5,7) : 0\\
(5,8) : 0    &    (5,9) : 0    &    (6,7) : 0    &    (6,8) : 1    &    (6,9) : 1    &    (7,8) : 0\\
(7,9) : 0    &    (8,9) : -1    &                       &                       &                       &                   \\
\hline
\end{tabular}
\end{center}
\end{enumerate}
\end{document}

