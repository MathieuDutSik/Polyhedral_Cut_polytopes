\documentclass[12pt]{article}
\usepackage{amsfonts, amsmath, latexsym, epsfig}
\usepackage{epsf}
\usepackage{url}
\title{Automorphism group of Lorentzian lattices}

\def\QuotS#1#2{\leavevmode\kern-.0em\raise.2ex\hbox{$#1$}\kern-.1em/\kern-.1em\lower.25ex\hbox{$#2$}}


%\usepackage{vmargin}
%\setpapersize{custom}{21cm}{29.7cm}
%\setmarginsrb{1.7cm}{1cm}{1.7cm}{3.5cm}{0pt}{0pt}{0pt}{0pt}
%marge gauche, marge haut, marge droite, marge bas.

\begin{document}
\newcommand{\R}{\ensuremath{\mathbb{R}}}
\newcommand{\N}{\ensuremath{\mathbb{N}}}
\newcommand{\Q}{\ensuremath{\mathbb{Q}}}
\newcommand{\C}{\ensuremath{\mathbb{C}}}
\newcommand{\Z}{\ensuremath{\mathbb{Z}}}
\newcommand{\T}{\ensuremath{\mathbb{T}}}
\newtheorem{proposition}{Proposition}
\newtheorem{theorem}{Theorem}
\newtheorem{corollary}{Corollary}
\newtheorem{lemma}{Lemma}
\newtheorem{problem}{Problem}
\newtheorem{conjecture}{Conjecture}
\newtheorem{claim}{Claim}
\newtheorem{remark}{Remark}
\newtheorem{definition}{Definition}
\newcommand{\qed}{\hfill $\Box$ }
\newcommand{\proof}{\noindent{\bf Proof.}\ \ }

Number of vertices $n=12$.

Adjacencies of Graph
\begin{enumerate}
\item vertex 1 adjacent to 2 3 12
\item vertex 2 adjacent to 1 4 11
\item vertex 3 adjacent to 1 4 5
\item vertex 4 adjacent to 2 3 6
\item vertex 5 adjacent to 3 6 7
\item vertex 6 adjacent to 4 5 8
\item vertex 7 adjacent to 5 8 9
\item vertex 8 adjacent to 6 7 10
\item vertex 9 adjacent to 7 10 11
\item vertex 10 adjacent to 8 9 12
\item vertex 11 adjacent to 2 9 12
\item vertex 12 adjacent to 1 10 11
\end{enumerate}
Size of automorphism group of the graph=24

Full group: $\vert Aut(polytope) \vert  =49152$

Restricted group: $\vert Aut(G) \times switch \vert = 49152$

Number of orbits for the full group : 6

List of orbits of facets for the full group:
Total number of orbits = 6
Total number of facets = 26452
\begin{enumerate}
\item Inequality 1 with incidence 1024 and stabilizer of size 1024. Orbit size is 
48 nature: 4-cycle inequality, C=[ 1, 2, 11, 12 ] F=[ 1, 2 ]
\begin{center}
\begin{tabular}{|c|c|c|c|c|c|}
\hline
(1,2) : -1    &    (1,3) : 0    &    (1,12) : 1    &    (2,4) : 0    &    (2,11) : 1    &    (3,4) : 0\\
(3,5) : 0    &    (4,6) : 0    &    (5,6) : 0    &    (5,7) : 0    &    (6,8) : 0    &    (7,8) : 0\\
(7,9) : 0    &    (8,10) : 0    &    (9,10) : 0    &    (9,11) : 0    &    (10,12) : 0    &    (11,12) : 1\\
\hline
\end{tabular}
\end{center}
\item Inequality 2 with incidence 1024 and stabilizer of size 4096. Orbit size is 
12 nature: edge inequality e=[ 1, 2 ]
\begin{center}
\begin{tabular}{|c|c|c|c|c|c|}
\hline
(1,2) : 1    &    (1,3) : 0    &    (1,12) : 0    &    (2,4) : 0    &    (2,11) : 0    &    (3,4) : 0\\
(3,5) : 0    &    (4,6) : 0    &    (5,6) : 0    &    (5,7) : 0    &    (6,8) : 0    &    (7,8) : 0\\
(7,9) : 0    &    (8,10) : 0    &    (9,10) : 0    &    (9,11) : 0    &    (10,12) : 0    &    (11,12) : 0\\
\hline
\end{tabular}
\end{center}
\item Inequality 3 with incidence 1024 and stabilizer of size 2048. Orbit size is 
24 nature: edge inequality e=[ 2, 4 ]
\begin{center}
\begin{tabular}{|c|c|c|c|c|c|}
\hline
(1,2) : 0    &    (1,3) : 0    &    (1,12) : 0    &    (2,4) : 1    &    (2,11) : 0    &    (3,4) : 0\\
(3,5) : 0    &    (4,6) : 0    &    (5,6) : 0    &    (5,7) : 0    &    (6,8) : 0    &    (7,8) : 0\\
(7,9) : 0    &    (8,10) : 0    &    (9,10) : 0    &    (9,11) : 0    &    (10,12) : 0    &    (11,12) : 0\\
\hline
\end{tabular}
\end{center}
\item Inequality 4 with incidence 224 and stabilizer of size 64. Orbit size is 
768 nature: 7-cycle inequality, C=[ 10, 12, 1, 3, 5, 7, 8 ] F=[ 10, 12 ]
\begin{center}
\begin{tabular}{|c|c|c|c|c|c|}
\hline
(1,2) : 0    &    (1,3) : 1    &    (1,12) : 1    &    (2,4) : 0    &    (2,11) : 0    &    (3,4) : 0\\
(3,5) : 1    &    (4,6) : 0    &    (5,6) : 0    &    (5,7) : 1    &    (6,8) : 0    &    (7,8) : 1\\
(7,9) : 0    &    (8,10) : 1    &    (9,10) : 0    &    (9,11) : 0    &    (10,12) : -1    &    (11,12) : 0\\
\hline
\end{tabular}
\end{center}
\item Inequality 5 with incidence 72 and stabilizer of size 48. Orbit size is 
1024 nature: 9-cycle inequality, C=[ 10, 12, 11, 2, 4, 3, 5, 7, 8 ] F=[ 10, 12 ]
\begin{center}
\begin{tabular}{|c|c|c|c|c|c|}
\hline
(1,2) : 0    &    (1,3) : 0    &    (1,12) : 0    &    (2,4) : 1    &    (2,11) : 1    &    (3,4) : 1\\
(3,5) : 1    &    (4,6) : 0    &    (5,6) : 0    &    (5,7) : 1    &    (6,8) : 0    &    (7,8) : 1\\
(7,9) : 0    &    (8,10) : 1    &    (9,10) : 0    &    (9,11) : 0    &    (10,12) : -1    &    (11,12) : 1\\
\hline
\end{tabular}
\end{center}
\item Inequality 6 with incidence 20 and stabilizer of size 2. Orbit size is 24576 nature: unknown
\begin{center}
\begin{tabular}{|c|c|c|c|c|c|}
\hline
(1,2) : 0    &    (1,3) : 2    &    (1,12) : -2    &    (2,4) : 1    &    (2,11) : 1    &    (3,4) : 1\\
(3,5) : 1    &    (4,6) : 2    &    (5,6) : -1    &    (5,7) : 2    &    (6,8) : 1    &    (7,8) : 1\\
(7,9) : 1    &    (8,10) : 2    &    (9,10) : -1    &    (9,11) : 2    &    (10,12) : 1    &    (11,12) : 1\\
\hline
\end{tabular}
\end{center}
\end{enumerate}
\end{document}

