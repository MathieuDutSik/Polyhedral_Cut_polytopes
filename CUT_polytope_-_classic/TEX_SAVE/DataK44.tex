\documentclass[12pt]{article}
\usepackage{amsfonts, amsmath, latexsym, epsfig}
\usepackage{epsf}
\usepackage{url}
\title{Automorphism group of Lorentzian lattices}

\def\QuotS#1#2{\leavevmode\kern-.0em\raise.2ex\hbox{$#1$}\kern-.1em/\kern-.1em\lower.25ex\hbox{$#2$}}


%\usepackage{vmargin}
%\setpapersize{custom}{21cm}{29.7cm}
%\setmarginsrb{1.7cm}{1cm}{1.7cm}{3.5cm}{0pt}{0pt}{0pt}{0pt}
%marge gauche, marge haut, marge droite, marge bas.

\begin{document}
\newcommand{\R}{\ensuremath{\mathbb{R}}}
\newcommand{\N}{\ensuremath{\mathbb{N}}}
\newcommand{\Q}{\ensuremath{\mathbb{Q}}}
\newcommand{\C}{\ensuremath{\mathbb{C}}}
\newcommand{\Z}{\ensuremath{\mathbb{Z}}}
\newcommand{\T}{\ensuremath{\mathbb{T}}}
\newtheorem{proposition}{Proposition}
\newtheorem{theorem}{Theorem}
\newtheorem{corollary}{Corollary}
\newtheorem{lemma}{Lemma}
\newtheorem{problem}{Problem}
\newtheorem{conjecture}{Conjecture}
\newtheorem{claim}{Claim}
\newtheorem{remark}{Remark}
\newtheorem{definition}{Definition}
\newcommand{\qed}{\hfill $\Box$ }
\newcommand{\proof}{\noindent{\bf Proof.}\ \ }

Number of vertices $n=8$.

Adjacencies of Graph
\begin{enumerate}
\item vertex 1 adjacent to 5 6 7 8
\item vertex 2 adjacent to 5 6 7 8
\item vertex 3 adjacent to 5 6 7 8
\item vertex 4 adjacent to 5 6 7 8
\item vertex 5 adjacent to 1 2 3 4
\item vertex 6 adjacent to 1 2 3 4
\item vertex 7 adjacent to 1 2 3 4
\item vertex 8 adjacent to 1 2 3 4
\end{enumerate}
Size of automorphism group of the graph=1152

Full group: $\vert Aut(polytope) \vert  =147456$

Restricted group: $\vert Aut(G) \times switch \vert = 147456$

Number of orbits for the full group : 4

List of orbits of facets for the full group:
Total number of orbits = 4
Total number of facets = 27968
\begin{enumerate}
\item Inequality 1 with incidence 64 and stabilizer of size 4608. Orbit size is 32
\begin{center}
\begin{tabular}{|c|c|c|c|c|c|}
\hline
(1,5) : 0&(1,6) : 0&(1,7) : 0&(1,8) : 1&(2,5) : 0&(2,6) : 0\\
(2,7) : 0&(2,8) : 0&(3,5) : 0&(3,6) : 0&(3,7) : 0&(3,8) : 0\\
(4,5) : 0&(4,6) : 0&(4,7) : 0&(4,8) : 0&               &               \\
\hline
\end{tabular}
\end{center}
\item Inequality 2 with incidence 64 and stabilizer of size 512. Orbit size is 288
\begin{center}
\begin{tabular}{|c|c|c|c|c|c|}
\hline
(1,5) : 0&(1,6) : 0&(1,7) : 1&(1,8) : 1&(2,5) : 0&(2,6) : 0\\
(2,7) : -1&(2,8) : 1&(3,5) : 0&(3,6) : 0&(3,7) : 0&(3,8) : 0\\
(4,5) : 0&(4,6) : 0&(4,7) : 0&(4,8) : 0&               &               \\
\hline
\end{tabular}
\end{center}
\item Inequality 3 with incidence 24 and stabilizer of size 8. Orbit size is 18432
\begin{center}
\begin{tabular}{|c|c|c|c|c|c|}
\hline
(1,5) : 1&(1,6) : 1&(1,7) : 0&(1,8) : -2&(2,5) : 1&(2,6) : -1\\
(2,7) : 0&(2,8) : 0&(3,5) : 1&(3,6) : 1&(3,7) : 1&(3,8) : 1\\
(4,5) : 1&(4,6) : 1&(4,7) : -1&(4,8) : 1&               &               \\
\hline
\end{tabular}
\end{center}
\item Inequality 4 with incidence 16 and stabilizer of size 16. Orbit size is 9216
\begin{center}
\begin{tabular}{|c|c|c|c|c|c|}
\hline
(1,5) : 2&(1,6) : 1&(1,7) : 1&(1,8) : -2&(2,5) : -2&(2,6) : 2\\
(2,7) : 1&(2,8) : -1&(3,5) : 1&(3,6) : 2&(3,7) : -2&(3,8) : 1\\
(4,5) : 1&(4,6) : 1&(4,7) : 2&(4,8) : 2&               &               \\
\hline
\end{tabular}
\end{center}
\end{enumerate}
\end{document}

